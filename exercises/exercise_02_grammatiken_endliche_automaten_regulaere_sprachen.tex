% Created 2025-12-31 Mi 14:01
% Intended LaTeX compiler: xelatex
\documentclass[a4paper,14pt,ngerman]{scrartcl}
\usepackage{graphicx}
\usepackage{longtable}
\usepackage{wrapfig}
\usepackage{rotating}
\usepackage[normalem]{ulem}
\usepackage{capt-of}
\usepackage{hyperref}
\usepackage{polyglossia}
\usepackage{polyglossia}
\setmainlanguage[variant=german]{german}
\usepackage{polyglossia}
\setdefaultlanguage{german}
\PolyglossiaSetup{german}{shorthands=on}
\AtBeginDocument{\let\babelprovide\relax}
\AtBeginDocument{\let\shorthandon\relax}
\AtBeginDocument{\let\shorthandoff\relax}
\setmainfont{Gentium Book Basic}
\setsansfont{Latin Modern Sans}
\setmonofont{Latin Modern Mono}
\usepackage{fancyhdr}
\usepackage{graphicx}
\usepackage{amsmath,amssymb}
\usepackage{xcolor}
\usepackage{xurl}
\usepackage{listings}
\usepackage{setspace}
\setstretch{1.3}
\author{Elias}
\date{\today}
\title{Übungsblatt 2}
\hypersetup{
 pdfauthor={Elias},
 pdftitle={Übungsblatt 2},
 pdfkeywords={},
 pdfsubject={},
 pdfcreator={Emacs 30.1 (Org mode 9.7.11)}, 
 pdflang={German}}
\begin{document}

\section{\textbf{Übung 2}}
\label{sec:orgae6ceff}
\subsection{\textbf{Grammatiken, endliche Automaten, reguläre Sprachen}}
\label{sec:orgd0fa655}
\subsubsection{\textbf{Aufgabe 1}}
\label{sec:org6c6bd17}

Welche Sprache \(L(G)\) wird durch die folgende Grammatik \(G\) erzeugt?
Ist die Grammatik eindeutig? Falls nein: lässt sich für diese Sprache
eine eindeutige Grammatik angeben? Von welchem Typ ist die Sprache?

\[\begin{aligned}
        G&=\left(\left\{S,A,B\right\},\left\{a,b,c\right\},P,S\right)\\
        P&=\left\{S \rightarrow aB \mid Ac, A \rightarrow ab, B \rightarrow bc\right\} 
\end{aligned}\]
\subsubsection{\textbf{Aufgabe 2}}
\label{sec:org6d6eee4}

Ist die Sprache
\(L = \left\{b^na^{2m} \mid n,m \in \mathbb{N}_0\right\}\) regulär?
Beantworten Sie die Frage vom Standpunkt einer erzeugenden Grammatik,
sowie von dem eines akzeptierenden Automaten.
\subsubsection{\textbf{Aufgabe 3}}
\label{sec:orgb8ad5da}

Konstruieren Sie einen DFA, der alle Worte über \(\{0,1\}\) erkennt, die
den String \(101\) \uline{nicht} als Teilwort enthalten.
\subsubsection{\textbf{Aufgabe 4}}
\label{sec:org6f0a90d}

Konstruieren Sie einen DFA, der die Menge aller natürlichen Zahlen
kongruent zu Null (mod 5) akzeptiert. Eingabe ist jeweils eine
natürliche Zahl in Dualdarstellung.
\subsubsection{\textbf{Aufgabe 5}}
\label{sec:orgc1e48da}

Geben Sie eine Typ-3 Grammatik an, die die durch folgenden Automaten
akzeptierte Sprache produziert. Beschreiben Sie die akzeptierte Sprache
verbal. Ermitteln sie den Minimalautomaten zu \(A\).

\[\begin{aligned}
        A&=\left(\left\{s_0,s_1,s_2,s_3,s_4\right\},\left\{0,1\right\},\delta,s_0,\left\{s_3,s_4\right\}\right)\\
        \delta &=\{(s_0,0,s_1),(s_0,1,s_2),(s_1,0,s_2),(s_1,1,s_3),(s_2,0,s_1),(s_2,1,s_3),\\
        &\qquad(s_3,0,s_1),(s_3,1,s_4),(s_4,0,s_2),(s_4,1,s_4)\}
\end{aligned}\]
\subsubsection{\textbf{Aufgabe 6}}
\label{sec:orgf8047b6}

Konstruieren Sie zu folgendem nichtdeterministischem endlichem Automaten
\[A=(\{p,q,r\}, \{a,b\}, \delta, p, \{r\})\] mit der mengenwertigen Übergangsfunktion \(\delta\)

\begin{center}
\begin{center}
\begin{tabular}{llll}
\(\delta\) & \(p\) & \(q\) & \(r\)\\
\hline
\(a\) & \(\{p,q\}\) & \(\{p,q,r\}\) & \(\{\}\)\\
\(b\) & \(\{r\}\) & \(\{\}\) & \(\{q,r\}\)\\
\end{tabular}
\end{center}
\end{center}

einen äquivalenten deterministischen Automaten.\\
Ermitteln Sie zu diesem DFA den Minimalautomaten.
\end{document}
