% Created 2025-12-31 Mi 14:01
% Intended LaTeX compiler: xelatex
\documentclass[a4paper,14pt,ngerman]{scrartcl}
\usepackage{graphicx}
\usepackage{longtable}
\usepackage{wrapfig}
\usepackage{rotating}
\usepackage[normalem]{ulem}
\usepackage{capt-of}
\usepackage{hyperref}
\usepackage{polyglossia}
\usepackage{polyglossia}
\setmainlanguage[variant=german]{german}
\usepackage{polyglossia}
\setdefaultlanguage{german}
\PolyglossiaSetup{german}{shorthands=on}
\AtBeginDocument{\let\babelprovide\relax}
\AtBeginDocument{\let\shorthandon\relax}
\AtBeginDocument{\let\shorthandoff\relax}
\setmainfont{Gentium Book Basic}
\setsansfont{Latin Modern Sans}
\setmonofont{Latin Modern Mono}
\usepackage{fancyhdr}
\usepackage{graphicx}
\usepackage{amsmath,amssymb}
\usepackage{xcolor}
\usepackage{xurl}
\usepackage{listings}
\usepackage{setspace}
\setstretch{1.3}
\author{Elias}
\date{\today}
\title{Übungsblatt 3}
\hypersetup{
 pdfauthor={Elias},
 pdftitle={Übungsblatt 3},
 pdfkeywords={},
 pdfsubject={},
 pdfcreator={Emacs 30.1 (Org mode 9.7.11)}, 
 pdflang={German}}
\begin{document}

\section{\textbf{Übung 3}}
\label{sec:org8ccd51c}
\subsection{\textbf{Kellerautomaten}}
\label{sec:org7be88d3}
\subsubsection{\textbf{Aufgabe 1}}
\label{sec:orgb4ce2b1}

Sind die folgenden Sprachen kontextfrei, bzw. sogar deterministisch
kontextfrei? Begründen Sie Ihre Antwort, gegebenenfalls durch Angabe
einer geeigneten Grammatik bzw. eines geeigneten Automaten.

\begin{enumerate}
\item \(L_1=\{a^ib^jc^k \mid i>j \textrm{ oder } j>k,\quad i,j,k\geq 0\}\)

\item \(L_2=\{ww \mid w\in \{a\}^*\}\)

\item \(L_3=\{ww \mid w\in \{a,b\}^*\}\)
\end{enumerate}
\subsubsection{\textbf{Aufgabe 2}}
\label{sec:orgc9cec8a}

Konstruieren Sie einen deterministischen PDA, der die Sprache\\
\(L=\{a^nb^m \mid m \leq 2n, \quad m,n\geq 1\}\) akzeptiert. Ist das
mittels leerem Keller möglich?
\subsubsection{\textbf{Aufgabe 3}}
\label{sec:orgf447778}

Konstruieren Sie einen deterministischen PDA, der die Sprache der
korrekt geklammerten Ausdrücke\\
\(L=\{w=xy \in \{[,]\}^* \mid \textrm{Anzahl}(w,[)=\textrm{Anzahl}(w,]),\)\\
und für jede Vorsilbe \(x\) gilt:
\(\textrm{Anzahl}(x,[)\geq\textrm{Anzahl}(x,])\}\)\\
akzeptiert.

Kann das durch Akzeptanz mit leerem Keller erreicht werden? Begründen
Sie!
\subsubsection{\textbf{Aufgabe 4}}
\label{sec:org175aca2}

Die Sprache \(L=\{a^ib^ia^kb^k \mid i,k\geq 1\}\) ist deterministisch
kontextfrei.\\
Lässt sich diese Sprache durch einen DPDA mit leerem Keller
akzeptieren?\\
Entwickeln Sie einen DPDA mit dem geeigneten Akzeptanzverhalten.
\end{document}
